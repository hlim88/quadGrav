\documentclass[a4paper,oneside,openany,11pt]{memoir}
\usepackage{geometry}
\geometry{
	inner=1.5cm, % Inner margin
	outer=2cm, % Outer margin
	bindingoffset=.5cm, % Binding offset
	top=3.5cm, % Top margin
	bottom=3.5cm, % Bottom margin
	%	showframe, % Uncomment to show how the type block is set on the page
}
\linespread{1.15}

\usepackage[english]{babel}	
\usepackage{xspace}
\usepackage{microtype}	
\usepackage[utf8]{inputenc}	% Allows for writing special charachters in the tex-file 

\usepackage{amsmath,amssymb,amsfonts,dsfont,bm} 	% Standard mathematics 
\allowdisplaybreaks 	%Allows pagebreak during multiline math enviroments
\usepackage{mathrsfs}
\usepackage[usenames,dvipsnames]{xcolor}
\usepackage{hyperref}
\usepackage{slashed}
\usepackage{youngtab}
\usepackage{physics}
\usepackage{tensor}


\usepackage{siunitx}
\sisetup{exponent-product = \cdot, 
	separate-uncertainty} 

%Bibliography
\usepackage[sort&compress, numbers, merge]{natbib}
\setlength{\bibsep}{0.0em}
\bibliographystyle{apsrev4-1}
\addto\captionsenglish{\renewcommand*{\bibname}{\LARGE References}}
\makeatletter
%\renewcommand{\@memb@bchap}{ \section*{\bibname} \bibmark \prebibhook}
\makeatother

%Figures and tables
\usepackage{graphicx}
\graphicspath{{./Figures/}}


\counterwithout{section}{chapter} % Include if sections should be taken at the top level
\counterwithout{figure}{chapter} %Removes chapter number on figures
\setsecnumdepth{subsection}
\numberwithin{equation}{section} %Denotes that equations should be numbered a section level


\newcommand{\aaron}[1]{{\color{OliveGreen} #1}}
\newcommand{\TODO}[1]{{\color{red}[}{\color{red}TODO:} {\color{blue}#1}{\color{red}]}}
\newcommand{\NOTE}[1]{{\color{blue}[#1]}}

%Table of contents
\renewcommand{\aftertoctitle}{\par\nobreak }
\maxtocdepth{subsection}


% % % % Document % % % % 
\begin{document}

\title{\HUGE Dynamical Black Holes and Gravitational Waves in Quadratic Gravity}

\author{Hyun Lim, Aaron Held\\(work in progress)}


\maketitle

\tableofcontents*

%=======================================================================================================

\section{Quadratic Gravity in the Strong-Gravity Regime}

Einstein's theory of General Relativity (GR) is built upon diffeomorphism symmetry, manifest in its formulation in terms of the Einstein-Hilbert action
\begin{align}
    S_{\text{GR}} &=
 \frac1{16\pi G_\text{N}}\int_x \sqrt{\text{det } g}\, (2\Lambda-R)\;,
\end{align}
where $G_N$ and $\Lambda$ denote the Newton coupling and cosmological
constant, respectively. It is a simple theory with only two free
parameters and describes all measured gravitational phenomena to
date. Apart from simplicity, a l\`a Occam's razor, there is no
fundamental principle in classical gravity that would explain the
absence of higher-order, diffeomorphism-invariant curvature terms.
The next (quadratic) order in curvature invariants, i.e., Quadratic
Gravity (QG) is given by
\begin{align}
    S_{\text{gravity}} &=
 S_\text{GR}
 + \int_x \sqrt{\text{det }g} \left(\alpha  R_{\mu\nu}R^{\mu\nu} + \beta  R^2\right)  \;. 
\end{align}
In spacetimes topologically equivalent to flat space the Gauss-Bonnet
topological invariant ensures that this is the most general action
at quadratic order. Kellog Stelle showed that, opposed to GR, QG
is perturbatively renormalizable as a quantum field theory
\cite{Stelle:1976gc}. Him, and many subsequent authors, also discuss
the appearence of additional massive ghost-like modes, one scalar
and one spin-2, which spoil the unitarity of this quantized theory
of gravity. Further, the theory can be motivated as a generic
infra-red limit of an effective field theory (EFT) treatment of the
quantization of gravity, see e.g. \cite{Donoghue:2012zc} for a
pedagogical review, and \cite{} for how this formulation avoids the
issue of unitarity. All subsequent (unresummed local) curvature
invariants will be suppressed by powers of the Planck scale.
\\
One might argue for the uniqueness of GR by the fact that it admits
a well-posed (numerical) evolution. In fact, as David R. Noakes
showed in \cite{Noakes:1983}, also the dynamics of quadratic gravity
can be formulated as a well-posed initial value problem.
We regard both these concepts of such crucial importance for all
what follows, that we will review them. Readers not interested in
a motivation for QG because of its well-posed numerical evolution
or for the absence of even higher orders because of effective field
theory, can safely skip Sec.~\ref{sec:EFT} and \ref{sec:well-posedness},
respectively.
\\

Experiments on solar-system scales only test the weak-gravity regime:
the existing constraints on higher-order modifications of GR are
therefore extremely weak. Submillimeter-tests using pendulums
constrain Yukawa-like corrections to the Newtonian potential that
arise from higher-derivative terms like $\alpha  R_{\mu\nu}R^{\mu\nu}$
and $\beta  R^2$. But, the constraints are as weak as $\alpha,\,\beta
< 10^{60}\sim 10^{70}$ \cite{Hoyle:2004cw, Kapner:2006si,
Calmet:2017rxl}. Further, they do not allow to distinguish between
the two couplings.
\\
The main motivation for this paper is to generate gravitational
wave templates from QG, to use experimental data from binary mergers
to constraint the QG-couplings $\alpha$ and $\beta$, see also
\cite{Berry:2011pb, Cao:2013osa} for a similar study in $f(R)$-gravity.
A simple comparison of typical curvature scales exemplifies how
vastly studies of the strong-gravity regime could improve these
bounds. For that, we compare the curvature at the surface of the
earth horizon with that of a solar-mass black hole, by use of the
Kretschmann scalar $K\sim M^2/r^6$, i.e.,
\begin{align}
	\frac{K_{\oplus\text{-surface}}}{K_{\odot\text{-horizon}}} = \frac{K_{\oplus\text{-surface}}}{K_{\oplus\text{-horizon}}}\;\frac{K_{\oplus\text{-horizon}}}{K_{\odot\text{-horizon}}} \approx 10^{-32}\;.
\end{align}

\subsection{Testing General Relativity}
aLIGO/VIRGO has detected gravitational waves (GWs) and these observations extend into the strong-field regime of gravity,
where the gravitational field is non-linear and dynamical, precisely where tests of general relativity(GR) are currently lacking. 
Strong field tests of GR have implications to a large areas in physics and astrophysics. For example, 
gravitational parity breaking modifies the geometry of spinning black holes (BHs) and the propagation of GWs in 
these backgrounds. Constrain such a departure from the Kerr geometry of GR in the strong field regime will place
constraints on the coupling constants of such theories.\TODO{Need references}

In this work, we are interested in dynamics and stability of BH in QG.
In literatures \TODO{Need references}, considerable works are done withimposing extra 
scalar field into the action such that (in the most general form)
\begin{align}
S &\equiv \int d^4 s \sqrt{-g} \Bigg\{ \frac{1}{16 \pi G_N} R + \alpha_1 f_1 (\vartheta) R^2 + \alpha_2 f_2 (\vartheta) R_{\mu \nu} R^{\mu \nu} 
    + \alpha_3 f_3 (\theta) R_{\mu \nu \lambda \sigma} R^{\mu \nu \lambda \sigma} \nonumber \\
    &+ \alpha_4 f_4 (\vartheta) R_{\mu \nu \lambda \sigma} (*R^{\mu \nu \lambda \sigma}) - \frac{\beta}{2} [ \nabla_a \vartheta \nabla^a \vartheta
    + 2 V(\vartheta) ] + \mathcal{L}_{matter} \Bigg\}
\end{align}
where $g$ stands for the determinant of the metric $g_{\mu \nu}$. $R, \, R_{\mu \nu}, \, R_{\mu \nu \lambda \sigma}$, 
and $*R_{\mu \nu \lambda \sigma}$ are the Ricci scalar, Ricci tensor, Riemann tensor and its dual respectively. 
$\vartheta$ is a dynamical scalar field, $f_i (\vartheta)$ are functionals of this field, $(\alpha_i , \beta)$ are coupling constants.
Coupling to a scalar field $\vartheta$ enables to make dynamical theory such as dynamical Chern-Simon (dCS) gravity 
and Einstein-dilaton-Gauss-Bonnet (EdGB) gravity as examples of QG theories.
\NOTE{What is the major difference (in motivation point of view) having (or not having) dynamical scalar field?}







\section{Review: Quadratic Gravity from an Effective-Field-Theory Quantization}
\label{sec:EFT}
\TODO{rewrite}
Field-theoretic attempts to quantize gravity have difficulties
retaining renormalizability. Without higher-derivative terms, General
Relativity is a non-renormalizable theory. The underlying reason
can be associated with the negative canonical dimension of the
Newton coupling, i.e., $[G_N]=-2$. The quantization of gravity is
not just an academic problem. If $G_N$ was of order one at accessible
energy scales, then the effects of delocalized quantum matter would
directly imply significant quantum-fluctuations in the associated
gravitational field. It is only the smallness of the gravitational
force, i.e., $G_N\sim 10^{-38}$ GeV, which hides these effects from
experimental observation. In fact, this is one particularly insightful
way of obtaining the Planck scale. The dimensionless Newton coupling
$g_N(\mu) = G_N\times \mu^2$ scales with a quadratic power-law in
the characteristic energy scale $\mu$. Since it is dimensionless
couplings which determine field-theoretic cross sections, gravitational
effects become important whenever $g_N(\mu)\approx 1$, i.e., at a
mass scale of $M_\text{Planck} = 10^{19}$ GeV -- the Planck scale.
\\

The most agnostic approach to the quantization of gravity is to
look at quantum gravity as an effective field theory (EFT). In
correspondence to, for instance, the Standard Model, this assumes
that \emph{all} operators allowed by symmetry are present at
$M_\text{Planck}$. Neglecting the index structure we can schematically
denote those (dimensionful) couplings and the corresponding curvature
invariants as $C_N\times R^N$. Since the underlying theory is not
known, \emph{all} EFT-couplings are assumed to be of order one at
the Planck-scale, i.e., $C_N(\mu=M_\text{Planck})\approx 1$. The
EFT draws its predictive power from canonical scaling. Since in
four dimensions the dimension of curvature is $[R]=2$, the associated
couplings have dimension $[C_N] = 2N - 4$. The corresponding
dimensionless couplings $c_N(\mu) = C_N / \mu^{2N-4}$ scale like
$c_N \sim \mu^{4-2N}$. Hence they are suppressed by
\begin{align}
    c_N(M_\text{exp}) \approx \left(\frac{M_\text{exp}}{M_\text{Planck}}\right)^{(2N-4)}\;.
\end{align}
    In an EFT one would thus conclude that all couplings $c_{N>2}$
    are suppressed by increasing powers of the enormously large
    Planck scale. This does not apply for the quadratic couplings
    $c_{N=2}$, which constitute so-called marginal couplings of the
    EFT. These are \emph{not} power-law suppressed and are hence
    expected to be of similar order as at the Planck-scale.
\\
    
The EFT for quantum gravity is valid below the Planck scale. It
might be possible to embedd it into a non-perturbatively renormalizable
quantum field theory of gravity solely defined by diffeomorphism
symmetry and a corresponding asymptotically safe fixed point. This
theory of Asymptotically Safe Gravity (ASG) was conjectured by
Steven Weinberg in 1976 \cite{Weinberg:1980gg}.  It is supported
by the mounting evidence around the corresponding Reuter fixed-point
\cite{Reuter:1996cp}, cf.~\cite{} for reviews. Since the Reuter
fixed-point is fully interacting, \emph{all} higher-order couplings
will be present at Planckian energies. Below the Planck scale the
EFT-description applies and one, again, expects the associated
low-energy effective theory to be governed by the four couplings
of QG, i.e., $G_N$, $\Lambda$, $\alpha$ and $\beta$. It can be added
here, that current approximations of the Reuter fixed-point indicate
that there are only three so-called relevant couplings. AS restores
the predictivity of a quantum field theory of gravity precisely by
non-perturbative relations, which express \emph{all} other (irrelevant)
couplings in terms of the three relevant ones. Hence, one of the
two QG couplings could follow as a prediction from AS, e.g., $\alpha
= \alpha (\beta,\,G_N,\,\Lambda)$.

\section{Review: Well-posed Initial Value Problem for Quadratic Gravity}
\label{sec:well-posedness}

\subsection{Equations of Motion in the Quadratic Gravity}

The equations of motion (eom) of QG are given by
\begin{align}
\label{eq:eom-QG}
	H_{\mu\nu}=\kappa\,G_{\mu\nu} + E_{\mu\nu}=\frac{1}{2}T_{\mu\nu}\;,
	\quad\quad
	\text{with: }\kappa = \frac{1}{16 \pi G_N}\;,
\end{align}
where $G_{\mu\nu} = R_{\mu\nu} -1/2 R g_{\mu\nu}$ is the usual Einstein tensor, which is supplemented by its quadratic-order counter-part
\begin{align}
	E_{\mu\nu} = &(\alpha-2\beta)\nabla_\mu\nabla_\nu R-\alpha \square R_{\mu\nu}-(\frac{1}{2}\alpha-2\beta)g_{\mu\nu}\square R
+2\alpha R^{\alpha\beta}R_{\mu\alpha\nu\beta}
	\notag\\
	&-2\beta R R_{\mu\nu}-\frac{1}{2}g_{\mu\nu}(\alpha R_{\alpha\beta}R^{\alpha\beta}-\beta R^2)\;.
\end{align}
QG propagates, besides the graviton, an additional massive scalar mode associated with the Ricci-scalar $R$ and a massive spin-2 mode corresponding to the traceless part of the Ricci-tensor $\widetilde{R}_{\mu\nu} = R_{\mu\nu} - \frac{1}{4}g_{\mu\nu}R$ \cite{Stelle:1977ry}. Knowing this, it is useful to split the equations of motion into a trace and a traceless part, i.e.,
\begin{align}
	\label{eq:eomForTraceRicci-1}
	\square R = &\frac{\kappa}{2(3\beta-2\alpha)}R\;,
	\textbf{\aaron{[have to check the factor of 2]}}
	\\
	\alpha\,\square \tilde{R}_{\mu\nu} = &-\kappa\,\tilde{R}_{\mu\nu} -(2\beta-\alpha) \left(\frac{\kappa}{8(3\beta-2\alpha)}g_{\mu\nu} -\nabla_\mu\nabla_\nu\right) R 
	\notag\\
	\label{eq:eomForTracelessRicci-1}
	&- (\alpha-2\beta) R\widetilde{R}_{\mu\nu}
	-2\alpha \left(R_{\mu\rho\nu\sigma} -\frac{1}{4} g_{\mu\nu}\, \widetilde{R}_{\rho\sigma}\right) \widetilde{R}^{\rho\sigma}
\end{align}
As anticipated, these equations are fourth order in derivatives.
Following \cite{Noakes:1983}, there are two more essential steps to cast the equations of motion into a well-posed IVP: (i) we employ harmonic coordinates to treat $g_{\mu\nu}$, $\widetilde{R}_{\mu\nu}$, and $R$ as independet variables, which reduces the system to contain only second-order derivatives; (ii) we use a differentiation procedure to diagonalize the resulting equations and put them in quasilinear form.

\subsection{Reduction to a second order system}

The Ricci tensor in a general coordinate system is given by
\begin{align}
	\label{eq:RicciTensorInTermsOfMetric}
	R_{\mu\nu}&=-\frac{1}{2}g^{\alpha\beta}\partial_{\alpha}\partial_{\beta}g_{\mu\nu}+Q_{\mu\nu}(g,\partial g)+\frac{1}{2}
\left(g_{\mu\beta}\partial_\nu F^\beta+g_{\nu\beta}\partial_\mu F^\beta\right),
	\\[0.5em]\notag
	\text{where }\quad
	Q^{\mu\nu}(g,\partial g)&=g^{\alpha\beta}\left(\Gamma_{\alpha\gamma}^{\mu}\partial_{\beta} g^{\nu\gamma}+\Gamma_{\alpha\gamma}^{\nu}\partial_{\beta} g^{\mu\gamma}-2\Gamma_{\alpha\beta}^{\gamma}\partial_{\gamma} g^{\nu\mu}\right)\;,
	\\[0.5em]\notag
	\text{and }\quad
	F^\alpha &= g^{\mu\nu}\Gamma_{\mu\nu}^\alpha = \frac{1}{\sqrt{-g}}\frac{\partial}{\partial x^\alpha}\bigg(\sqrt{-g}g^{\alpha\beta}\bigg)\;.
\end{align}
Harmonic coordinates are defined by $F^{\mu} = 0$. This choice is advantagous because it allows to reduces the expression of the Ricci tensor in terms of the metric into a quasilinear form, that is the last term in Eq.~\eqref{eq:RicciTensorInTermsOfMetric} vanishes in harmonic coordinates. Hence, 
\begin{align}
	\label{eq:EOMforMetricInHarmonicCoords}
	-\frac{1}{2}g^{\alpha\beta}\partial_{\alpha}\partial_{\beta}g_{\mu\nu}+Q_{\mu\nu}(g,\partial g) = R_{\mu\nu} = \widetilde{R}_{\mu\nu}
+\frac{1}{4}g_{\mu\nu}R\;.
\end{align}
Adding this to the equations of motion and treating $g_{\mu\nu}$, $\widetilde{R}_{\mu\nu}$, and $R$ as independent variables reduces the system to second order. Notice that Eq.~\eqref{eq:EOMforMetricInHarmonicCoords} and therefore all following equations only hold in harmonic coordinates. The second order set of equations reads
\begin{align}
	\frac{1}{2}g^{\alpha\beta}\partial_{\alpha}\partial_{\beta}g_{\mu\nu} = &Q_{\mu\nu}(g,\partial g)-\widetilde{R}_{\mu\nu}
-\frac{1}{4}g_{\mu\nu}R\;, 	\label{eom:metric}
	\\
	\square R = &\frac{\kappa}{2(3\beta-2\alpha)}R\;, \label{eom:ricci_scalar}
	\\
	\alpha\,\square \tilde{R}_{\mu\nu} = &
	(2 \beta - \alpha) \nabla_a \nabla_b R - \frac{1}{16 \pi} \tilde{R}_{ab} - \frac{2 \beta - \alpha}{128\pi(3\beta - 2 \alpha)} g_{ab} R \nonumber \\
&+(\alpha - 2 \beta)\left[g^{cd}g^{mn} g_{mn,cd} - 2 Q(g,\partial g) \right]\tilde{R}_{ab} \nonumber \\
&+\frac{\alpha}{4} g_{ab} \left[ g^{cd}g_{mn,cd} - 2 Q_{mn} (g,\partial g) \right] g^{om}g^{pn} \tilde{R}_{op} \nonumber \\
&- \alpha \bigg[ g_{cd,ab} + g_{ab,cd} - g_{cb,ad} - g_{ad,cb} +4 g_{mn} \Gamma\indices{^m_{c[b}} \Gamma\indices{^n_{d]a}} \bigg] \tilde{R}^{cd}. 	\label{eom:ricci_tensor}
\end{align}
Here we have written $2\beta\,R$ and $R_{\mu\rho\nu\sigma} -\frac{1}{4} g_{\mu\nu}\, \widetilde{R}_{\rho\sigma}$ in Eq.~\eqref{eq:eomForTracelessRicci-1} in terms of the metric (using again harmonic coordinates). 
\textbf{\aaron{[motivation not really clear to me]}}
This set of equations is now only of second order, but the last equation is still not quasilinear.

\subsection{Diagonalization to a quasilinear hyperbolic system}
We can diagonalize the equations to a quasilinear system by introducing additional variables $V_\mu = \partial_\mu R$ and $h_{\mu\nu\alpha} = g_{\mu\nu,\alpha}$ and adding derivatives of the first two (already quasilinear) equations to the system. We obtain
\begin{align}
	-\frac{1}{2}g^{\eta\delta}g_{\mu\nu,\eta\delta}=
	&-Q_{\mu\nu}(g,\partial g)+\widetilde{R}_{\mu\nu} + \frac{1}{4}g_{\mu\nu}R,
	\\
	g^{\alpha\beta}\partial_{\alpha}\partial_{\beta}R=
	&\frac{\kappa}{2(3\beta-2\alpha)}R
	\\
	-\frac{1}{2}g^{\eta\delta}h_{\mu\nu\gamma,\eta\delta}=
	&\frac{1}{2}g^{\eta\delta}_{\,\,\,\,,\gamma}\,h_{\mu\nu\eta,\delta}-Q_{\mu\nu,\gamma}(g,\partial g, c)+\widetilde{R}_{\mu\nu,\gamma}
	+\frac{1}{4}g_{\mu\nu,\gamma}R+\frac{1}{4}g_{\mu\nu}V_\gamma,
	\\
	g^{\alpha\beta}\partial_{\alpha}\partial_{\beta}V_\gamma=
	&\frac{\kappa}{2(3\beta-2\alpha)}V_\gamma,
	\\
	\alpha \square \widetilde R_{\mu\nu}=
	&(2\beta-\alpha) \nabla_{\mu}V_\nu
	-\kappa\,\widetilde{R}_{\mu\nu}
	-\frac{\kappa(2\beta-\alpha)}{8(3\beta-2\alpha)}g_{\mu\nu}R 
	\\\notag
	&-(\alpha-2\beta)\left(-g^{\alpha\beta}g^{\sigma\rho}h_{\sigma\rho\alpha,\beta}+2Q(g,\partial g)\right)\widetilde{R}_{\mu\nu}
	-\frac{\alpha}{8} g_{\mu\nu}\left(-g^{\alpha\beta}h_{\rho\sigma\alpha,\beta}+2Q_{\rho\sigma}(g,\partial g)\right)\widetilde{R}^{\rho\sigma}
	\\\notag
	&-\alpha \left(h_{\rho\sigma\mu,\nu} +h_{\mu\nu\rho,\sigma}-h_{\rho\nu\mu,\sigma} -h_{\mu\sigma\rho,\nu}+2g_{\alpha\beta}(\Gamma^{\alpha}_{\rho\nu}\Gamma^{\beta}_{\mu\sigma}-\Gamma_{\rho\sigma}^{\alpha}\Gamma_{\mu\nu}^{\beta})\right)\widetilde{R}^{\rho\sigma}.
\end{align}
the final well-posed form of the evolution equations.

\section{Recast Evolution Equations into 3+1 Form}

In this section, we reduce our systems of equations in 3+1 form. There are several reasons to use 3+1 form:

1. We can have a system of equation in terms of first order in time and second order in space i.e. better form
as a numerical aspects (Note that this is not always true but at least for me this is true)

2. Easily adapt gauge choice for BH rather than using excision techniques i.e. do not need to solve elliptic problems.

3. Constraints are treated as evolution vars so we can monitor is easily.

We are following usual 3+1 variables such that
\begin{align}
\label{eqn:3p1}
ds^2 = -\alpha^2 dt^2 + \gamma_{ij} (dx^i + \beta^i dt)(dx^j + \beta^j dt)
\end{align}
where $\alpha$ is lapse, $\beta^i$ is shift, $\gamma_{ij}$ is induced metric on spatial hypersurface.
Thus, the metric can be expressed as
\begin{align}
g_{ab} = g_{ij}\gamma^i_a \gamma^j_b - n_a n_b
\end{align}
where $n^a$ is the covariant normal to the spacelike hypersurface.

Using these defintions, it is obvious to split Ricci tensor
\begin{align}
\label{ricci3p1}
R_{ab} n^a n^b &= (\partial_\bot K + D_i D^i \alpha)/\alpha - K_{ij} K^{ij} \\
R_{ab} \gamma^a_i \gamma^b_j &= R_{ij} + K K_{ij} - 2 K_{ik} K^{k}_j - (\partial_\bot K_{ij})/\alpha
-(D_i D_j \alpha)/\alpha \\
R_{ab} \gamma^a_i n^b &= - D_j K^j_i + D_i K
\end{align}
Here, we define $\partial_\bot \equiv \partial_t - \pounds_\beta$. For the Ricci scalar
\begin{align}
^{(4)}R = R + K_{ij} K^{ij} + K^2 - 2 (\partial_\bot K) /\alpha - 2 (D_i D^i) \alpha / \alpha
\end{align}

In previous section, we derived the EOM in second order forms 
(Eqns.~\ref{eom:metric},~\ref{eom:ricci_scalar}, and~\ref{eom:ricci_tensor} ).

From Eqn.~\ref{eom:metric}. we obtain (Note that this is usual equations from metric)
\begin{align}
\partial_\bot \gamma_{ij} &= - 2 \alpha K_{ij} \\
\partial_\bot K_{ij} &= \alpha (R_{ij} - 2 K_{ik}K^{k}_j) - D_i D_j \alpha
\end{align}
There is a gauge freedom for lapse and shift. We will determine these variables later.
Now consider Eqns.~\ref{eom:ricci_scalar} and~\ref{eom:ricci_tensor}.
We first consider LHSs of these equations. 
\begin{align}
\Box R &= \nabla_a \nabla^a R 
=g\indices{^b_a} \nabla_b(g^{ac} \nabla_c R) \nonumber \\
&= (\gamma\indices{^b_a} - n_a n^b)\nabla_b [(\gamma^{ac} - n^a n^c) \nabla_c R]
\end{align} 
where $\nabla_a$ is usual covariant derivative.
Note that the derivatives of the physical field are being projected into and normal to the spacelike
hypersurfaces. Using this, we will define a new variable $\hat{R} = - n^a \nabla_a R$.
Without whole detailed derivations (If you want, I can type it up but basic ideas are pretty standard in 
3+1 decomposition...), we can obtain
\begin{align}
\nabla_a \nabla^a R = n^a \nabla_a \hat{R} + \frac{1}{\alpha} D_a (\alpha D^a R) - K \hat{R}
\end{align}
where $D_a$ is 3D covariant derivative (or gradient) which lives on spacelike hypersurface.
Combining above result with RHS of Eqn.~\ref{eom:ricci_scalar} provides two first order in time equations
for $\hat{R}$ and R which are given by
\begin{align}
n^a \nabla_a R &= - \hat{R}  \\
n^a \nabla_a \hat{R} &= -  \frac{1}{\alpha} D_a (\alpha D^a R) + K \hat{R} + \frac{1}{36 \pi (3\beta_c - 2 \alpha_c)} R
\end{align}
Note that here we invoke $\alpha_c$ and $\beta_c$ which are slightly different notations than previous sections
to avoid conflict between lapse ($\alpha$) and shift ($\beta$).
We can re-write above equations also (just for keeping consistency)
\begin{align}
\partial_\bot R &= - \alpha \hat{R}  \\
\partial_\bot \hat{R} &= - D_a (\alpha D^a R) + \alpha K \hat{R} + \frac{\alpha \kappa}{2 (3\beta_c - 2 \alpha_c)} R
\end{align}
where we use the notation $n^a \nabla_a f = \frac{1}{\alpha} (\partial_t - \beta^i \partial_i)f$ for $f$ a scalar

Similarly, for Eqn.~\ref{eom:ricci_tensor} (now we deal with tensor not a scalar), we have
\begin{align}
\Box \tilde{R}_{ab} &= \nabla_c \nabla^c \tilde{R}_{ab} \nonumber \\
&= (\gamma\indices{^d_c} - n_c n^d) \nabla_d [(\gamma\indices{^{ce}} - n^c n^e)] \nabla_e \tilde{R}_{ab} \nonumber \\
&= \gamma\indices{^d_c} \nabla_d (\gamma^{ce} \nabla_e \tilde{R}_{ab}) - n_c n^d \nabla_d (\gamma^{ce} \nabla_e \tilde{R}_{ab})
- \gamma\indices{^d_c} \nabla_d (n^c n^e \nabla_e \tilde{R}_{ab}) + n_c n^d \nabla_d ( n^c n^e \nabla_e \tilde{R}_{ab}) \nonumber \\
\end{align}
We define $\tilde{V}_{ab} = - n^c \nabla_c \tilde{R}_{ab}$. We argue that this can be considered 
also as the time derivative (or velocity) as traceless of Ricci tensor (or say tensor flow) like in previous
Ricci scalar case. Thus, we will have
\begin{align}
\nabla_c \nabla^c \tilde{R}_{ab} = n^c \nabla_c \tilde{V}_{ab} + \frac{1}{\alpha} D_c (\alpha D^c \tilde{R}_{ab}) - K \tilde{V}_{ab}
\end{align}
Now consider RHS of Eqn.~\ref{eom:ricci_tensor}
\begin{align}
Y_{ab}= &(2 \beta_c - \alpha_c) \nabla_a \nabla_b R - \frac{1}{16 \pi} \tilde{R}_{ab} - \frac{2 \beta_c - \alpha_c}{128\pi(3\beta_c - 2 \alpha_c)} g_{ab} R \nonumber \\
&+(\alpha_c - 2 \beta_c)\left[g^{cd}g^{mn} g_{mn,cd} - 2 Q(g,\partial g) \right]\tilde{R}_{ab} \nonumber \\
&+\frac{\alpha_c}{4} g_{ab} \left[ g^{cd}g_{mn,cd} - 2 Q_{mn} (g,\partial g) \right] g^{om}g^{pn} \tilde{R}_{op} \nonumber \\
&- \alpha_c \bigg[ g_{cd,ab} + g_{ab,cd} - g_{cb,ad} - g_{ad,cb} +4 g_{mn} \Gamma\indices{^m_{c[b}} \Gamma\indices{^n_{d]a}} \bigg] \tilde{R}^{cd} .
\end{align}
where we use $Y_{ab}$ just for convenience purpose. In $Y_{ab}$ most of terms are okay (i.e. no involving second derivatives) 
and most of terms are already decomposed like using Ricci tensors/scalar and metric. So, we split the $Y_{ab}$ into two pieces
\begin{align}
Y_{ab} = Y^{np}_{ab} + \Delta_c \big[(2 \beta_c - \alpha_c) \nabla_a \nabla_b R - \alpha_c (g_{cd,ab} + g_{ab,cd} - g_{cb,ad} - g_{ad,cb}) \tilde{R}^{cd} \big]
&= Y^{np}_{ab} + \Delta_c Y^{p}_{ab}
\end{align}
where $\Delta_c$ is coupling constant that occurs higher order derivatives and $Y^{np}_{ab}$ contains all remaining terms.
Thus, we may choose small value of $\Delta_c$ to control how quadratic terms in the theory effects on dynamics of BH.
We also use decompositions of Ricci tensors, Ricci scalar, and Christoffel symbols for $Y^{np}_{ab}$ \TODO{check this too}

Using all of these, we obtain
\begin{align}
n^a \nabla_a \tilde{R}_{ab} &= - \tilde{V}_{ab}  \\
n^a \nabla_a \tilde{V}_{ab} &= -  \frac{1}{\alpha} D_a (\alpha D^a \tilde{R}_{ab} ) + K \tilde{V}_{ab} + Y^{np}_{ab} + \Delta_c Y^{p}_{ab}
\end{align} 
\TODO{Does it make sense?}

\subsection{BSSN formulation}
So far, we have
\begin{align}
n^a \nabla_a R &= - \hat{R}  \\
n^a \nabla_a \hat{R} &= -  \frac{1}{\alpha} D_a (\alpha D^a R) + K \hat{R} + \frac{1}{36 \pi (3\beta_c - 2 \alpha_c)} R\\
n^a \nabla_a \tilde{R}_{ab} &= - \tilde{V}_{ab}  \\
n^a \nabla_a \tilde{V}_{ab} &= -  \frac{1}{\alpha} D_a (\alpha D^a \tilde{R}_{ab} ) + K \tilde{V}_{ab} + Y^{np}_{ab} + \Delta_c Y^{p}_{ab}
\end{align}
(Note that I omit the equations from metric because it will be same as usual standard GR). 
These are the usual 3+1 decomposition (or ADM decomposition) which has been shown to 
be weakly hyperbolic. Here we recast again these sets of equation in terms of BSSN form.
We will follow usual conventions for BSSN variable \TODO{Define vars.. but state here}.

After some manipulation, we have
\begin{align}
\label{eqn:bssn:inter1}
D_a ( \alpha D^a R) = \alpha \chi \left[\tilde{\gamma}^{ij} \partial_i \partial_j R + \tilde{\gamma}^{ij} (\partial_i  \ln \alpha) \partial_j R - \tilde{\Gamma}^i \partial_i R - \frac{1}{2} \tilde{\gamma}^{ij} \partial_i R \partial_j \ln \chi \right]
\end{align}

\begin{align}
\label{eqn:bssn:inter1}
D_a ( \alpha D^a \tilde{R}_{ab}) = \partial_i \alpha \left(\partial_i \tilde{R}^{ab} - \frac{3}{2\chi} \tilde{R}^{ab} \partial_a \chi \right) + \alpha (D^a D_a )_{\textrm{BSSN}} \tilde{R}^{ab}
\end{align}

Thus, we have
\begin{align}
n^a \nabla_a R &= - \hat{R}  \\
n^a \nabla_a \hat{R} &= -  \alpha \chi \left[\tilde{\gamma}^{ij} \partial_i \partial_j R + \tilde{\gamma}^{ij} (\partial_i  \ln \alpha) \partial_j R - \tilde{\Gamma}^i \partial_i R - \frac{1}{2} \tilde{\gamma}^{ij} \partial_i R \partial_j \ln \chi \right] \nonumber \\
& + K \hat{R} + \frac{1}{36 \pi (3\beta_c - 2 \alpha_c)} R\\
n^a \nabla_a \tilde{R}_{ab} &= - \tilde{V}_{ab}  \\
n^a \nabla_a \tilde{V}_{ab} &= -  \frac{1}{\alpha} \partial_i \alpha \left(\partial_i \tilde{R}^{ab} - \frac{3}{2\chi} \tilde{R}^{ab} \partial_a \chi \right) - (D^a D_a )_{\textrm{BSSN}} \tilde{R}^{ab} \nonumber \\
& + K \tilde{V}_{ab} + Y^{np}_{ab} + \Delta_c Y^{p}_{ab}
\end{align}


\subsection{Hyperbolicity of the System}
\TODO{Characteristic analysis is required}

\subsection{Gauge Choice and Full Evolution System}
\TODO{add discussions on $1+\log$ slicing and $\Gamma$-driver}


\section{Initial Data}

First, we use some exact solutions for Einstein's equations 
that are relevant to our case. We are using this to check 
the evolution system and the code that we have developed.

Minkowski space in the Cartesian coordinates
\begin{align}
\label{eqn:id:mink}
\alpha = 1 \\
\beta^i = 0 \\
\gamma_{ij} = \delta_{ij} \\
det(\gamma_{ij}) = 1  \\
K_{ij} = 0 \\
\bar{\gamma}_{ij} = \delta_{ij} \\
\phi = 0 \\
\bar{A}_{ij}
\end{align}

\subsection{Elementary Black Hole Solution}

In QG, we can still use element BHs to check 
dynamical stability. 

\subsubsection{Schwarzschild Black Hole}
Schwarzschild solution satisfies the Einstein's equation $R_{ab}=0$ i.e. Schwarzschild solution should
satisfy QG. We rewrite Schwarzschild solution in terms of BSSN variables such that
For example, Schwarzschild solution in spherical type Kerr-Schild coordinates
\begin{align}
\alpha &= \sqrt{\frac{r}{r+2M}} \\
\beta^r &= \frac{2M}{r+2M} \\
\beta_r &=\frac{2M}{r} \\
\beta^\theta &= \beta^\varphi = 0 \\
K_{ij} &= \textrm{diag} \left[ -\frac{2M(r+M)}{\sqrt{r^5 (r+2M)}} , 2M \sqrt{\frac{r}{r+2M}}, K_{\theta \theta} \sin^2 \theta \right]
\end{align}
Schwarzschild solution in Cartesian type Kerr-Schild coordinate
\begin{align}
\alpha &= \sqrt{\frac{r}{r+2M}} \\
\beta^i &= \frac{2M}{r} \frac{x^i}{r+2M} \\
\beta_i &=\frac{2M x_i}{r^2} \\
K_{ij} &= \-\frac{2M}{r^4} \sqrt{\frac{r}{r+2M}} \left[ \left( \frac{M}{r}+2 \right) x_i x_j - r^2 \delta_{ij} \right]
\end{align}
where $x^i = (x,y,z)$ which is usual spatial Cartesian coordinate. In both cases, we can see lapse is regular at the horizon.

\subsection{Black Holes in QG}
Here we describe our BH initial data. \NOTE{more descriptions?}

\subsection{Binary Black Hole Initial Data}


\subsubsection{Puncture Method}

\NOTE{Maybe solve elliptic equations or just find puncture like ID}

Under usual 3+1 decomposition \NOTE{Still same for QG?}, the constraint equations are (in vacua)
\begin{align}
D_j K\indices{^j_i} - D_i K &= 0 \\
R + K^2 - K_{ij} K^{ij} &= 0
\end{align}
The spatial metric $\gamma_{ij}$, the extrinsic curvature $K_{ij}$, and any matter field should 
satisfy the constraints. Thus, we have to specify $(\gamma_{ij}, K_{ij})$ on some initial 
spatial slice $\Sigma$ that are compatible with the constraint equations. These fields can 
then be used as initial data for a dynamical evolution obtained by solving the evolution equation.


\section{Gravitational Wave Extractions for QG}
Here, we calculate the $\Psi_4$ to extract the gravitational wave information. To do that, we first define tetrad. 
There are lots of possible ways to do this but we will try to follow the way I know 
(a way is in \texttt{hyperGHSF} code). We define the timelike member of the tetrad to be the 
normal to our spacelike hypersurfaces. The remaining three are then constructed 
via a Gram-Schmidt procedure from a set of three independent vectors living on the 
hypersurfaces. Our demand for them seem to be only that in the asymptotically flat limit, 
we recover something akin to the usual unit vectors of spherical coordinates. 

Indeed, we start with a version of them
\begin{align}
u^a &= (0,x,y,z) \\
v^a &= (0,xz,yz,-x^2 - y^2) \\
w^a &= (0,-y,x,0)
\end{align}
and then using the 3-metric, $\gamma_{ij}$ ,orthonormalize them with respect to it. In particular, we define new orthonormal spacelike vectors
\begin{align}
^{(1)}e^i &= \frac{u^i}{||u||} \\
^{(2)}e^i &= \frac{v^i -  \bra{ ^{(1)}e}\ket{v} \, ^{(1)}e^i}{||v -  \bra{ ^{(1)}e}\ket{v} \, ^{(1)}e ||} \\
^{(3)}e^i &= \frac{w^i -  \bra{ ^{(1)}e}\ket{w} \, ^{(1)}e^i -  \bra{ ^{(2)}e}\ket{w} \, ^{(2)}e^i}{||w -  \bra{ ^{(1)}e}\ket{w} \, ^{(1)}e - \bra{ ^{(2)}e}\ket{w} \, ^{(2)}e ||} 
\end{align}
where we are defining the inner product and the norm as
\begin{align}
\bra{u}\ket{v} \equiv \gamma_{ij} u^i v^j \\
|| u || \equiv \sqrt{\bra{u}\ket{u}}
\end{align}
with these, we construct a null tetrad according to
\begin{align}
l^a &= \frac{1}{2} (n^a + \, ^{(1)}e^a) \\
\tilde{n}^a &= \frac{1}{2} (n^a - \, ^{(1)}e^a) \\
m^a &= \frac{1}{2} (^{(2)}e^a + i \, ^{(3)}e^a) \\
\bar{m}^a &=  \frac{1}{2} (^{(2)}e^a - i \, ^{(3)}e^a)
\end{align}
where, because we are running out of letters, the usual null vector $n^a$ has been written with a tilde to distinguish if from the normal 
to the foliation. Then we can compute the relevant complex Penrose scalar $\Psi4$ such that
\begin{align}
\Psi_4 &= C_{abcd} \tilde{n}^a \bar{m}^b \tilde{n}^c \bar{m}^d \\
           &= C_{abcd} \tilde{n}^a \tilde{n}^c \left[\frac{1}{2} \{ ^{(2)}e^b \, ^{(2)}e^d - ^{(3)}e^b \, ^{(3)}e^d  \} + i \, ^{(2)}e^b \, ^{(3)}e^d \right] 
\end{align}
Now we must decompose this with respect to an assumed spacelike hypersurface. As usual, 
we define the normal to the hypersurface as $n_a$, the metric on the hypersurface as $\gamma_{ij}$ 
and the extrinsic curvature as $K_{ij}$.
\NOTE{Here is subtly. If we do notneed to consider additional higher derivatives of curvature into this manner, we may use same procedure as Einstein GR but not sure}


\bibliography{References}
	
\end{document}



