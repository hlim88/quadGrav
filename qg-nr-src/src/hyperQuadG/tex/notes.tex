\documentclass[letter,prd,aps,floatfix,superscriptaddress]{revtex4}

\usepackage{amssymb,amsmath,graphicx}
%\usepackage{epsfig}
\usepackage{hyperref}
\usepackage[usenames,dvipsnames]{color}
\usepackage{array}
\makeatletter
\renewcommand*{\p@section}{\S\,}
\renewcommand*{\p@subsection}{\S\,}
\renewcommand{\sectionautorefname}{\S}
\makeatother

%
%  These Macros are taken from the AAS TeX macro package version 4.0.
%  Include this file in your LaTeX source only if you are not using
%  the AAS TeX macro package and need to resolve the macro definitions
%  in the BibTeX entries returned by the ADS abstract service.
%
%  For more information on the AASTeX macro package, please see the URL
%	http://www.aas.org/publications/aastex.html
%  For more information about ADS abstract server, please see the URL
%	http://adswww.harvard.edu/ads_abstracts.html
%

% Abbreviations for journals.  The object here is to provide authors
% with convenient shorthands for the most "popular" (often-cited)
% journals; the author can use these markup tags without being concerned
% about the exact form of the journal abbreviation, or its formatting.
% It is up to the keeper of the macros to make sure the macros expand
% to the proper text.  If macro package writers agree to all use the
% same TeX command name, authors only have to remember one thing, and
% the style file will take care of editorial preferences.  This also
% applies when a single journal decides to revamp its abbreviating
% scheme, as happened with the ApJ (Abt 1991).

\def\jnl@style{\it}
%commente par Seb
\def\aaref@jnl#1{{\jnl@style#1}}
%ref remplace par aaref pour eviter conflit...

\def\aaref@jnl#1{{\jnl@style#1}}

\def\aj{\aaref@jnl{AJ}}                   % Astronomical Journal
\def\araa{\aaref@jnl{ARA\&A}}             % Annual Review of Astron and Astrophys
\def\apj{\aaref@jnl{ApJ}}                 % Astrophysical Journal
\def\apjl{\aaref@jnl{ApJ}}                % Astrophysical Journal, Letters
\def\apjs{\aaref@jnl{ApJS}}               % Astrophysical Journal, Supplement
\def\ao{\aaref@jnl{Appl.~Opt.}}           % Applied Optics
\def\apss{\aaref@jnl{Ap\&SS}}             % Astrophysics and Space Science
\def\aap{\aaref@jnl{A\&A}}                % Astronomy and Astrophysics
\def\aapr{\aaref@jnl{A\&A~Rev.}}          % Astronomy and Astrophysics Reviews
\def\aaps{\aaref@jnl{A\&AS}}              % Astronomy and Astrophysics, Supplement
\def\azh{\aaref@jnl{AZh}}                 % Astronomicheskii Zhurnal
\def\baas{\aaref@jnl{BAAS}}               % Bulletin of the AAS
\def\jrasc{\aaref@jnl{JRASC}}             % Journal of the RAS of Canada
\def\memras{\aaref@jnl{MmRAS}}            % Memoirs of the RAS
\def\mnras{\aaref@jnl{MNRAS}}             % Monthly Notices of the RAS
\def\pra{\aaref@jnl{Phys.~Rev.~A}}        % Physical Review A: General Physics
\def\prb{\aaref@jnl{Phys.~Rev.~B}}        % Physical Review B: Solid State
\def\prc{\aaref@jnl{Phys.~Rev.~C}}        % Physical Review C
\def\prd{\aaref@jnl{Phys.~Rev.~D}}        % Physical Review D
\def\pre{\aaref@jnl{Phys.~Rev.~E}}        % Physical Review E
\def\prl{\aaref@jnl{Phys.~Rev.~Lett.}}    % Physical Review Letters
\def\pasp{\aaref@jnl{PASP}}               % Publications of the ASP
\def\pasj{\aaref@jnl{PASJ}}               % Publications of the ASJ
\def\qjras{\aaref@jnl{QJRAS}}             % Quarterly Journal of the RAS
\def\skytel{\aaref@jnl{S\&T}}             % Sky and Telescope
\def\solphys{\aaref@jnl{Sol.~Phys.}}      % Solar Physics
\def\sovast{\aaref@jnl{Soviet~Ast.}}      % Soviet Astronomy
\def\ssr{\aaref@jnl{Space~Sci.~Rev.}}     % Space Science Reviews
\def\zap{\aaref@jnl{ZAp}}                 % Zeitschrift fuer Astrophysik
\def\nat{\aaref@jnl{Nature}}              % Nature
\def\iaucirc{\aaref@jnl{IAU~Circ.}}       % IAU Cirulars
\def\aplett{\aaref@jnl{Astrophys.~Lett.}} % Astrophysics Letters
\def\apspr{\aaref@jnl{Astrophys.~Space~Phys.~Res.}}
                % Astrophysics Space Physics Research
\def\bain{\aaref@jnl{Bull.~Astron.~Inst.~Netherlands}} 
                % Bulletin Astronomical Institute of the Netherlands
\def\fcp{\aaref@jnl{Fund.~Cosmic~Phys.}}  % Fundamental Cosmic Physics
\def\gca{\aaref@jnl{Geochim.~Cosmochim.~Acta}}   % Geochimica Cosmochimica Acta
\def\grl{\aaref@jnl{Geophys.~Res.~Lett.}} % Geophysics Research Letters
\def\jcp{\aaref@jnl{J.~Chem.~Phys.}}      % Journal of Chemical Physics
\def\jgr{\aaref@jnl{J.~Geophys.~Res.}}    % Journal of Geophysics Research
\def\jqsrt{\aaref@jnl{J.~Quant.~Spec.~Radiat.~Transf.}}
                % Journal of Quantitiative Spectroscopy and Radiative Transfer
\def\memsai{\aaref@jnl{Mem.~Soc.~Astron.~Italiana}}
                % Mem. Societa Astronomica Italiana
\def\nphysa{\aaref@jnl{Nucl.~Phys.~A}}   % Nuclear Physics A
\def\physrep{\aaref@jnl{Phys.~Rep.}}   % Physics Reports
\def\physscr{\aaref@jnl{Phys.~Scr}}   % Physica Scripta
\def\planss{\aaref@jnl{Planet.~Space~Sci.}}   % Planetary Space Science
\def\procspie{\aaref@jnl{Proc.~SPIE}}   % Proceedings of the SPIE

\let\astap=\aap
\let\apjlett=\apjl
\let\apjsupp=\apjs
\let\applopt=\ao


\newcommand{\Alfven}{Alfv\`en}
\def\LL#1{{\textcolor{green}{\bf LL: #1}}}
\def\SLL#1{{\textcolor{red}{\bf SLL: #1}}}
\def\CP#1{{\textcolor{blue}{\bf CP: #1}}}
\def\EOC#1{{\textcolor{cyan}{[\bf EOC: #1]}}}
\def\LC#1{{\textcolor{magenta}{[\bf LC: #1]}}}
\def\DN#1{{\textcolor{Plum}{[\bf DN: #1]}}}
\def\ewh#1{{\textcolor{Emerald}{[\bf ewh: #1]}}}


\def\rmd{{\rm d}}
\def\p{\partial}

\font\bfgreek=cmmib10
\font\bfGreek=cmb10

\newcommand{\had}{{\sc had}}
\newcommand{\etal}{{\it et al.{}}}
\def\bbU{{\hbox{\bfgreek\char'125}}}
\def\bba{{\hbox{\bfgreek\char'141}}}
\def\bbb{{\hbox{\bfgreek\char'142}}}
\def\bbc{{\hbox{\bfgreek\char'143}}}
\def\bbd{{\hbox{\bfgreek\char'144}}}
\def\bbe{{\hbox{\bfgreek\char'145}}}
\def\bbf{{\hbox{\bfgreek\char'146}}}
\def\bbg{{\hbox{\bfgreek\char'147}}}
\def\bbh{{\hbox{\bfgreek\char'150}}}
\def\bbi{{\hbox{\bfgreek\char'151}}}
\def\bbj{{\hbox{\bfgreek\char'152}}}
\def\bbk{{\hbox{\bfgreek\char'153}}}
\def\bbl{{\hbox{\bfgreek\char'154}}}
\def\bbm{{\hbox{\bfgreek\char'155}}}
\def\bbn{{\hbox{\bfgreek\char'156}}}
\def\bbo{{\hbox{\bfgreek\char'157}}}
\def\bbp{{\hbox{\bfgreek\char'160}}}
\def\bbq{{\hbox{\bfgreek\char'161}}}
\def\bbr{{\hbox{\bfgreek\char'162}}}
\def\bbs{{\hbox{\bfgreek\char'163}}}
\def\bbt{{\hbox{\bfgreek\char'164}}}
\def\bbu{{\hbox{\bfgreek\char'165}}}
\def\bbv{{\hbox{\bfgreek\char'166}}}
\def\bbw{{\hbox{\bfgreek\char'167}}}
\def\bbx{{\hbox{\bfgreek\char'170}}}
\def\bby{{\hbox{\bfgreek\char'171}}}
\def\bbz{{\hbox{\bfgreek\char'172}}}

%%%%%%%%%%%%%%%%%%%%%%%%%%%%%%%%%%%%%%%%%%%%%%%%%%%%%%%%%%%%%%%%%%%
%
%   B E G I N   D O C U M E N T
%
%%%%%%%%%%%%%%%%%%%%%%%%%%%%%%%%%%%%%%%%%%%%%%%%%%%%%%%%%%%%%%%%%%%%
\begin{document}

\date{\today}
\title{Notes about dilaton theory}

\maketitle

%\tableofcontents


%%%%%%%%%%%%%%%%%%%%%%%%%%%%%%%%%%%%%%%%%%%%%%%%%%%%%%%%%%%%%%%%%%%%

\section{Evolution equations}

Let us start with an action given by a Lagrangian in the physical
or Jordan frame defined on a string metric $g^{st}_{ab}$
({\bf is this the full Lagrangian? see for instance~\cite{1994GReGr..26.1171D}} 
{\ewh{It depends on what you are asking.  No, it is not the ``full" Lagrangian 
of low energy string theory.  But even that is a truncation of the string 
worldsheet action to consider only tree level (as opposed to higher order 
loop effects).  Even considering different string theories, the truncated 
actions (or Lagrangians) are different between type I, IIA, IIB, heterotic and 
so on.  But that is probably not what you are asking nor is that relevant to 
what we want to do.  What you have is (mostly) the correct Lagrangian for a 
subsector of low energy heterotic string theory in which some 
of the fields have been set to zero.  The reason it is mostly correct is 
that the $\alpha_0$ does not belong.  It's insertion usually comes at the 
stage that the Einstein frame is used and an association is made between that
Lagrangian (in the Einstein frame) and between other closely related  
theories.  But what you have is correct provided what you are interested in 
is a parametrized version of Einstein-Maxwell-dilaton theory.}})

\begin{eqnarray}
  {\cal L}_{st} =e^{-2 \alpha_0 \varphi} \left[ R_{st} + 4 \left( \nabla \varphi \right)^2 - F^2 \right]
\end{eqnarray}
that can be converted, by a conformal transformation
 $g_{ab} = e^{-2 \alpha_0 \varphi} g^{st}_{ab}$, into a Lagrangian in the Einstein frame 
\begin{eqnarray}
  {\cal L} = R - 2 \left( \nabla \varphi \right)^2 - 2 V(\varphi^2)- e^{-2 \alpha_0 \varphi} F^2
\end{eqnarray}
where we have included a massive potential $V(\varphi^2) = m^2 \varphi^2$.

{\bf We will assume that this is the correct Lagrangian in the Einstein frame from which we will derive the equations of motion} (see for instance~\cite{1993stqg.conf...55H,2008PhRvD..77d4034G}). By varying the action with respect to $g^{ab}$ we obtain the following equations
\begin{eqnarray}
  R_{ab} - \frac{1}{2} g_{ab} R &=& 2 T^{\varphi}_{ab} 
        + 2 e^{-2 \alpha_0 \varphi} T^{em}_{ab} \\
  T^{\varphi}_{ab}  &=& \nabla_a \varphi \nabla_b \varphi 
  - \frac{1}{2} g_{ab} \left[ \nabla_c \phi \nabla^c \varphi + V(\varphi^2) \right] \\
    T^{em}_{ab}  &=& F_{ac} {F_b}^c - \frac{1}{4} g_{ab} F_{cd} F^{cd} 
\end{eqnarray}

Notice that we can write this in terms of the Einstein equations
($G_{ab}=8 \pi T_{ab}$) with a standard stress-energy tensor given by
\begin{eqnarray}
  T_{ab} = \frac{1}{4 \pi} \left( T^{\varphi}_{ab}  
        + e^{-2 \alpha_0 \varphi} T^{em}_{ab} \right)
\end{eqnarray}
Just for completeness, these equations can be written in the Ricci formulation in a way consistent with Eric's equations, namely 
\begin{eqnarray}
  R_{ab} =  2 \nabla_a \varphi \nabla_b \varphi + g_{ab} V(\varphi^2)  
   + 2 e^{-2 \alpha_0 \varphi} 
   \left( F_{ac} {F_b}^c - \frac{1}{4} g_{ab} F_{cd} F^{cd} \right)
\end{eqnarray}


For the other equations we can use the Lagrange equations
\begin{eqnarray}
  \frac{\partial {\cal L}}{\partial \lambda} 
  -\partial_a \left[ \frac{{\cal L}}{\partial (\partial_a \lambda)} \right] = 0
\end{eqnarray}
The part of the Lagrangian containing scalar field terms is
\begin{eqnarray}
  {\cal L}_{\phi} = - 2 (\nabla \varphi)^2 - 2 V(\varphi^2) - e^{-2 \alpha_0 \varphi} F_{cd} F^{cd}
\end{eqnarray}
so, the final evolution equations for the scalar field are just
\begin{eqnarray}
  g^{ab} \nabla_a \nabla_b \varphi =  \frac{\partial V}{\partial \varphi^2} \varphi - \frac{\alpha_0}{2} e^{-2 \alpha_0 \varphi} F_{cd} F^{cd}
\end{eqnarray}
which can be written in terms of the old equations by substituting
$\frac{\partial V}{\partial \varphi^2} \varphi \rightarrow  \frac{\partial V}{\partial \varphi^2} \varphi - \frac{\alpha_0}{2} e^{-2 \alpha_0 \varphi} F^2$. Notice that we can expand this term by using
$F^2 = 2 (B^2 - E^2)$.

We can write down explicitly the KG equations in general
\begin{eqnarray}
  \partial_t \varphi &=& \beta^k \partial_k \varphi - \alpha \Pi
\\
  \partial_t \Pi &=& \beta^k \partial_k \Pi 
   +  \alpha \left[ -\gamma^{ij} \nabla_i \nabla_j \varphi + \Pi trK + V' \varphi - \frac{\alpha_0}{2} e^{-2 \alpha_0 \varphi} F^2\right]
   - \gamma^{ij} \nabla_i \varphi \nabla_j \alpha
\end{eqnarray}

where we have defined $V' \equiv dV/d \varphi^2$. We could rewrite easily this system for a conformal decomposition like the BSSN one as
\begin{eqnarray}
  \partial_t \phi &=& \beta^k \partial_k \varphi - \alpha \Pi
\\
  \partial_t \Pi &=& \beta^k \partial_k \Pi 
   +  \alpha \left[ -\chi \tilde{\gamma}^{ij} \partial_i \partial_j \varphi + \chi \tilde{\Gamma}^k \partial_k \varphi + \frac{1}{2} \tilde{\gamma}^{ij} \partial_i \varphi \partial_j \chi  + \Pi trK + V' \varphi
   - \frac{\alpha_0}{2} e^{-2 \alpha_0 \varphi} F^2 \right] 
\nonumber \\
&-& \chi \tilde{\gamma}^{ij} \partial_i \varphi \partial_j \alpha
\end{eqnarray}




Finally, the lagrangian containing the EM field is 
\begin{eqnarray}
  {\cal L}_{em} = - e^{-2 \alpha_0 \varphi} F_{cd} F^{cd}
\end{eqnarray}
where $F_{ab} = \partial_a A_b - \partial_b A_a$. The evolution equations
by varying $A_a$ are
\begin{eqnarray}
  \nabla_a \left(  e^{-2 \alpha_0 \varphi} F^{ab} \right) = 0
\end{eqnarray}
which can be written in terms of the standard EM fields with a current
\begin{eqnarray}
  \nabla_a  F^{ab}  = 2 \alpha_0 F^{ab} \nabla_a \varphi = -I^b
\end{eqnarray}
We can introduce the constraint damping fields, so that the Maxwell equations will be written as
\begin{eqnarray}
  \nabla_a  (F^{ab} + g^{ab} \psi)  &=& -I^b + \kappa n^b \psi\\
    \nabla_a  ({}^*F^{ab} + g^{ab} \phi)  &=& \kappa n^b \phi
\end{eqnarray}
Therefore, we only need to compute the projections of the 4-current, namely the 3-current and the electric charge density,
namely
\begin{eqnarray}
 J_i &\equiv& I_i = -2 \alpha_0 {F^a}_i \partial_a \varphi 
\\ 
q &\equiv& -n_b I^b = 2 \alpha_0 E^i \partial_i \varphi = -2 \alpha \alpha_0  F^{a0} \partial_a \varphi
\end{eqnarray}

\bibliographystyle{utphys}
\bibliography{notes}


\end{document}
