\def\E{\!\perp\!\! E}  
\def\B{\!\perp\!\! B}  
\def\F{\!\perp\!\! F}  
\def\H{\!\perp\!\! H}  
\def\HE{\!\perp\!\! {^{H\!}E} }   
\def\HB{{^{H\!}B}}   

\centerline{\bf BBH in low energy string theory}

\bigskip

Begin with the action for low energy (heterotic) string theory in 4 
dimensions:  
$$
S = \int d^4 x \sqrt{-g^{\rm str}} \, e^{-2\phi} \left[ R + \Lambda + 4\bigl( \nabla \phi \bigr)^2 - F^2 - {1\over12} H^2 \right] 
$$
where this is with respect to the string metric, $g^{\rm str}_{ab}$.  This is 
the metric that strings will couple to.  From the perspective of GR, this 
action replaces the standard Hilbert action ($\int R/16\pi G$).  It is derived
from the vanishing of the appropriate beta functions at tree level.  The 
real scalar field, $\phi$, is the dilaton, $F_{ab}$ is a $U(1)$ gauge field 
derivable from a potential, e.g. $F=d{\tilde A}$, and $H_{abc}$ is a 3-form also derivable from 
a potential, $B$, namely $H = dB - {\tilde A}\wedge F$.  (The latter results in 
$dH = -F\wedge F$.)  The constant $\Lambda$ may as well be thought of as a 
cosmological constant though it will be set to zero as we want asymptotically
flat boundary conditions at infinity.  Finally, $R$ is the standard Ricci
scalar built from the string metric.  The equations that follow from this 
action are not the ones that will be used, but we give them nonetheless
$$\eqalign{ 
0 & = R_{ab} + 2 \nabla_a \nabla_b \phi - 2 F_{ac} F_{b}{}^{c} - {1\over 4} H_{acd} H_b{}^{cd} \cr  
0 & = \nabla^a \bigl( e^{-2\phi} F_{ba} \bigr) + {1\over12} \, e^{-2\phi} \, H_{bcd} F^{cd} \cr 
0 & = \nabla^a \bigl( e^{-2\phi} H_{abc} \bigr) \cr 
0 & = 4 \nabla^2 \phi - 4 \bigl( \nabla \phi \bigr)^2 + \Lambda + R - F^2 - {1\over12} H^2 \cr 
}$$

What will actually be done is to perform a conformal transformation to the 
Einstein metric.  This is closer in spirit to what numerical relativity is 
usually done in.  Note that the above fields are an extension of pure 
Einstein gravity.  We will treat them as if they were matter fields after the 
conformal transformation.  However, we could, in fact, try to add matter to 
the original low energy string action.  For example, we could 
incorporate yet another scalar field, this time complex and charged under 
a different $U(1)$, call it $f$.  One might include this additional matter 
piece to the above action and allow the conformal transformation to 
appropriately couple it through the dilaton, $\phi$.  The resulting action,
again with respect to the string metric, would be  
$$
S = \int d^4 x \sqrt{-g^{\rm str}} \, \left\{ e^{-2\phi} \Bigl[ R + \Lambda + \bigl( \nabla \phi \bigr)^2 - F^2 - {1\over12} H^2 \Bigr] - a_0 (D \psi) (D \psi)^{*} - {a_1 \over 4}\, f^2 \right\}   
$$
where $a_0$ and $a_1$ are real coupling constants that are arbitrary at the 
moment and $D_a \psi \equiv \nabla_a \psi - i e A_a \psi$ is the metric 
and gauge covariant derivative of the complex scalar field $\psi$.  The charge,
$e$, is the coupling between the scalar field, $\psi=\psi_1 + i \psi_2$ and 
the gauge potential, $A$.  Of course, we have  
$f_{ab} = \nabla_a A_b - \nabla_b A_a$.   

On performing the conformal transformation, the action becomes, now with 
respect to the Einstein metric, $g_{ab} = e^{-2\phi} g^{\rm str}_{ab}$, 
$$
S = \int d^4 x \sqrt{-g} \, \Bigl[ R - 2 \bigl( \nabla \phi \bigr)^2 + \Lambda e^{2\phi} - e^{-2\phi} F^2 - {1\over12} e^{-4\phi} H^2 - a_0 \, e^{2\phi} \, (D \psi) (D \psi)^{*} - {a_1 \over 4}\, f^2 \Bigr]   
$$
where, it bears repeating, all derivatives, contractions and curvature pieces
are now performed or calculated with the Einstein metric, $g_{ab}$.  
The resulting equations of motion are (dropping the extra matter 
for the time being, i.e. setting $a_0=a_1=0$)  
$$\eqalign{ 
R_{ab} & = 2 \nabla_a \phi \nabla_b \phi - {1\over2} \Lambda e^{2\phi} \, g_{ab} + 2 e^{-2\phi} \Bigl[ F_{ac} F_{b}{}^{c} - {1\over4} \, g_{ab} \, F^2 \Bigr] + {1\over4} e^{-4\phi} \Bigl[ H_{acd} H_{b}{}^{cd} - {1\over3} \, g_{ab} \, H^2 \Bigr] \cr 
\nabla^a \nabla_a \phi & = - {1\over 2} \, \Lambda \, e^{2\phi} - {1\over2} \, e^{-2\phi} \, F^2 - {1\over12} \, e^{-4\phi} \, H^2 \cr 
\nabla^b \bigl( e^{-2\phi} F_{ba} \bigr) & = {1\over12} \, e^{-4\phi} \, H_{abc} F^{bc} \cr 
\nabla_a \bigl( e^{-4\phi} \, H^{abc} \bigr) & = 0 \cr 
}$$
A couple of comments are in order here.  Note that the cosmological constant now
acts like a real potential for the dilaton.  This could be generalized to 
an arbitrary potential, $V(\phi)$, such that the dilaton has a mass term
and higher order self-interaction pieces beyond an exponential coupling to 
$\Lambda$.  Supposedly, when supersymmetry gets broken, such mass terms for the
dilaton should arise.  We will not include these here.  Indeed, we will 
set $\Lambda$ 
to zero.  

In some treatments, the exponential coupling on $F^2$ is replaced 
with $e^{-2\alpha_0\phi}$ where $\alpha_0$ is allowed to 
be a nonnegative, real constant completely unrelated to the lapse, $\alpha$.  
If $H$ is set to zero, the resulting theory
is sometimes called Einstein-Maxwell-dilaton theory and we have a family
of theories parametrized by the constant $\alpha_0$.  In particular, 
$\alpha_0=0$ is an uncoupled real scalar field together with the 
Einstein-Maxwell equations, $\alpha_0=\sqrt{3}$ is Kaluza-Klein theory, 
and $\alpha_0=1$ is, of course, a sector of low energy string 
theory.  Such a generalization is certainly valid for $H=0$.  If the $H$ 
field is nonzero, it is not clear what the self-consistent generalization 
for Kaluza-Klein is.  Nonetheless, we will still include the constant 
$\alpha_0$ in the $e^{-4\phi}$ coupling on the $H^2$ term thus allowing for 
a continuous family of theories parametrized by $\alpha_0$.  The simplest way 
to do this is to let $\phi \rightarrow \phi' = \alpha_0 \phi$ in the action
(in the Einstein frame) and then reset the coefficient on the kinetic term 
for the scalar field to $2$.  

There are black hole solutions in these theories in various limits.  We will 
discuss them later, but 
the key point here is that for rotating solutions that generalize Kerr beyond
vacuum GR to ``vacuum" low energy string theory, the $H$ field is a necessary
ingredient.  The so-called Sen or Kerr-Sen solution has nontrival metric, 
dilaton, Maxwell, and three-form fields.  To set any one of the latter three
to zero reduces the others to zero as well if what you want is a rotating 
(nonspherical) black hole.    

We now decompose the equations with respect 
to a spacelike slice of the spacetime with normal $n^a$ and projector 
$h_{ab} = g_{ab}+n_a n_b$.  Doing this leads to the standard Einstein 
equations plus the evolution and constraint equations for the ``matter" 
pieces introduced.  In particular, the latter lead to 
$$\eqalign{
%
n^a \nabla_a \phi & = \Pi \cr 
%
n^a \nabla_a \Pi & = {1\over\alpha} D^a\bigl( \alpha D_a \phi\bigr) + K \Pi + {\alpha_0\over 2} \, e^{-2\alpha_0\phi} \Bigl[ \, \F_{ab} \F^{ab} - 2 \E_a \E^a \Bigr] 
\cr 
& \qquad 
    + {\alpha_0\over12} \, e^{-4\alpha_0\phi} \, \Bigl[ \, \H_{abc} \H^{abc} - 3 \HE_{ab} \HE^{ab} \Bigr] \cr 
%
n^a \nabla_a \E_b & = - {1\over\alpha} D^a\bigl( \alpha \F_{ab} \bigr) + 2\alpha_0 D^a\phi \F_{ab} + 2\alpha_0 \Pi \E_b - \E_a K_b{}^a + K \E_b 
\cr 
& \qquad 
    + {1\over12} \, e^{-2\alpha_0\phi} \Bigl[ \, \H_{bcd} \F^{cd} - 2 \HE_{bc} \E^c \Bigr] 
\cr 
& \qquad 
    + n_b \, \Bigr[ {1\over\alpha} \, D^a\bigl( \alpha \! \E_a \bigr) - 2\alpha_0 \E_a D^a\phi - {1\over12} \, e^{-2\alpha_0\phi} \, \HE_{cd} \F^{cd} \Bigr] \cr 
%
n^a \nabla_a \HE_{bc} & = {1\over\alpha} D^a \bigl( \alpha \H_{abc} \bigr) - 4 \alpha_0 D^a \phi \H_{abc} + 4\alpha_0 \Pi \HE_{bc} + K \HE_{bc} + 2 \HE_{a[b} K_{c]}{}^a 
\cr
& \qquad 
    - {2\over\alpha} \Bigl[ D^a\bigl( \alpha \HE_{a[b} \bigr) n_{c]} - 4\alpha_0 D^a \phi \HE_{a[b} n_{c]} \Bigr] \cr 
%
}$$
where the first two equations result from the dilaton equation, the third 
equation includes the expansion of the above EM equation and the final one is 
the expansion of the above equation for the $H$ field.  We use the following 
definitions 
$$\eqalign{ 
\F_{ab} & = h_a{}^c h_b{}^d F_{cd} \cr 
\E_a & = h_a{}^b F_{bc} n^c \cr 
\H_{abc} & = h_a{}^d h_b{}^e h_c{}^f H_{def} \cr 
\HE_{ab} & = h_a{}^d h_b{}^e H_{def} n^f \cr 
}$$
These are all purely spatial tensors orthogonal in all indices to the normal
$n^a$.  Note that the two ``$\E$" fields, $\E_a$ and $\HE_{ab}$, are 
``electric" parts of their respective antisymmetric tensors, $F$ and $H$.  
Needless to say, we can also identify the complete spatial projections of 
$F$ and $H$ as the ``magnetic" parts of the same tensors for which there  
should also be associated equations of motion.  Said another way, in addition 
to the equations above for $F_{ab}$ and $H_{abc}$, we must include
the equations for their duals.  By virtue of their definitions, we also 
have $dF=0$ and $dH = -F\wedge F$.  These become  
$$\eqalign{ 
%
n^a\nabla_a \bigl[ \epsilon^{bcd}\F_{cd} \bigr] 
  & = {1\over\alpha} D^a\bigl[ \alpha \, \epsilon^{abc} \E_c \bigr] - {1\over2} \, K_a{}^b \, \epsilon^{acd} \F_{cd} + {1\over2} \, K \epsilon^{bcd} \F_{cd} + {1\over2} \, n^b \, {1\over\alpha} D^a \bigl( \alpha \, \epsilon^{acd} \F_{cd} \bigr) \cr 
%
n^a\nabla_a \bigl( \epsilon_{bcd} \H^{bcd} \bigr) 
  & = {3 \over \alpha} \, D_a \bigl( \alpha \, \epsilon^{abc} \HE_{bc} \bigr) + \epsilon_{abc} \H^{abc} \, K + 4 \E_a \bigl( \epsilon^{abc} \F_{bc} \bigr) \cr 
%
}$$
where $\epsilon_{abc} = n^d \epsilon_{dabc}$.  On using the definitions
$$\eqalign{ 
B_a & = {1\over2} \, \epsilon_{abc} \F^{bc} \cr 
\HB & = {1\over6} \, \epsilon_{abc} \H^{abc} \cr 
}$$ 
we can simplify the equations a bit further.  

The decomponsed stress tensor, or better yet, the decomposed right hand side 
of the Einstein equations, becomes 
$$\eqalign{ 
%
G_{ab} & = ``8\pi G" T_{ab} \cr  
  & = h_a{}^i h_b{}^j {\perp\!T}_{ij} - 2 n_{(a} h_{b)}{}^i \, j_i + n_a n_b \, \rho \cr 
  & = 2 D_a \phi \, D_b \phi + 2 e^{-2\alpha_0\phi} \, \bigl( \F_{ac} \F_b{}^c - \E_a \E_b  \bigr) + {1\over 4} \, e^{-4\alpha_0\phi} \, \bigl[ \H_{acd} \H_b{}^{cd} - 2 \HE_b{}^c \HE_{ac} \bigr] 
\cr
 & \qquad 
    - h_{ab} \, \Bigl[ D^c\phi \, D_c\phi - \Pi^2 + {1\over2} \, e^{-2\alpha_0\phi} \, \bigl( \F_{cd} \F^{cd} - 2 \E_c \E^c \bigr) + {1\over24} \, e^{-4\alpha_0\phi} \, \bigl( \H_{cde} \H^{cde} - 3 \HE_{cd} \HE^{cd} \bigr) \Bigr]  
\cr 
 & \qquad 
    + 2 \, n_{[a} h_{b]}{}^i \, \Bigl[ 2 \Pi D_i\phi - 2 e^{-2\alpha_0\phi} \F_i{}^c \E_c - e^{-4\alpha_0\phi} \H_{icd} \HE^{cd} \Bigr] 
\cr 
 & \qquad 
    + n_a n_b \, \Bigl[ D_c\phi D^c \phi + \Pi^2 + {1\over2} e^{-2\alpha_0\phi} \bigl( \F_{cd} \F^{cd} - 2 \E_c \E^c \bigr) 
\cr 
 & \qquad\qquad\qquad 
    + {1\over 24} e^{-4\alpha_0\phi} \bigl( \H_{cde} \H^{cde} - 3 \HE_{cd} \HE^{cd} \bigr) \, \Bigr]  \cr 
%
}$$
where ${\perp\!\! T}_{ij}$, $j_i$ and $\rho$ are the projected stress 
tensor, the momentum flux and the energy density, respectively.   

Using these definitions, we can write 
$$\eqalign{ 
%
\partial_t \phi & = \beta^i \partial_i \phi + \alpha \Pi \cr 
%
\partial_t \Pi & = \beta^a \partial_i \Pi + D^i\bigl( \alpha D_i \phi \bigr) + \alpha K \Pi + \alpha_0 \alpha e^{-2\alpha_0\phi} \bigl[ \B_i\B^i - \E_i\E^i \bigr]
\cr 
  & \qquad 
    + {\alpha_0 \over 2} \, \alpha e^{-4\alpha_0\phi} \bigl[ \HB^2 - \HE_i \HE_i \bigr] \cr  
%
\partial_t \E_a & = \beta^i \partial_i \E_a - 2 \alpha \Pi \E_a - \E_b \partial_a \beta^b - K \E_a - {1\over6} \, e^{-2\alpha_0\phi} \, \bigl[ \HB \B_a - \epsilon_{abc} \, \E^b \HE^c \bigr] \cr 
%
0 & = D^b \bigl( \alpha \E_b \bigr) - 2\alpha_0 \alpha \E_b D^b \phi - {1\over6} \, \alpha e^{2\alpha_0\phi} \, \B_b \HE^b \cr 
%
\partial_t \bigl( \epsilon_{bcd} \HE^d \bigr) 
  & = \beta^i \partial_i \bigl( \epsilon_{bcd} \HE^d \bigr) - \bigl( \epsilon_{acd} \HE^d \bigr) \partial_b \beta^a + \bigl( \epsilon_{abd} \HE^d \bigr) \partial_c \beta^a 
\cr 
& \qquad 
      + 2 \bigl( \epsilon_{abd} \HE^d \bigr) K_c{}^a - 2 \bigl( \epsilon_{acd} \HE^d \bigr) K_b{}^a + K \bigl( \epsilon_{bcd} \HE^d \bigr) 
\cr 
& \qquad 
      - e^{-2\alpha_0\phi} \epsilon_{abc} D^a \bigl( \alpha \HB \bigr) 
      + 4\alpha_0 \alpha \epsilon_{abc} \HB D^a \phi \cr
%
0 & = \epsilon_{cad} D^a \bigl( \alpha \HE^a \bigr) - 2\alpha_0 e^{-2\alpha_0\phi} \, \epsilon_{cad} \HE^d D^a \phi \cr 
%
\partial_t \B^b & = \beta^i \partial_i \beta^b - \B^i \partial_i \beta^b + D_a \bigl( \alpha \epsilon^{abc} \E_c \bigr) + \alpha K \B^b \cr 
%
0 & = D_i\bigl( \alpha \B^i \bigr) \cr
%
\partial_t \HB & = \beta^i \partial_i \HB + D_a \bigl( \alpha \HE^a \bigr) + \alpha K \HB + {2\over3} \, \E_a \B^a \cr  
%
}$$


\vfil\eject

\bigskip
\centerline{\bf BBH in Einstein-Maxwell-Dilaton}
\bigskip

For the time being, consider just the reduced theory in which $H_{abc}$, 
or the axion, is zero.  Include the parameter $\alpha_0$ in order to 
parametrize a family of theories.  Include as well a mass term for the 
dilaton, though it's unclear whether there are exact BH solutions that 
incorporate the mass.  The action is 
$$
S = \int d^4 x \sqrt{-g} \, \Bigl[ R - 2 \bigl( \nabla \phi \bigr)^2 - 2 V(\phi) - e^{-2\alpha_0 \phi} F^2 \, \Bigr]   
$$
where to give the dilaton a mass, the potential $V(\phi) = m^2 \phi^2$.  On 
varying the action, the equations of motion become 
$$\eqalign{ 
R_{ab} & = 2 \nabla_a \phi \nabla_b \phi + g_{ab} V + 2 e^{-2\alpha_0\phi} \Bigl[ F_{ac} F_{b}{}^{c} - {1\over4} \, g_{ab} \, F^2 \Bigr] \cr 
\nabla^a \nabla_a \phi & = {1\over2} {\partial V \over \partial \phi} - {\alpha_0 \over2} \, e^{-2\alpha_0\phi} \, F^2 \cr
\nabla^a \bigl( e^{-2\alpha_0 \phi} F_{ab} \bigr) & = 0 \cr 
}$$
These are supplemented with the identity $\nabla_{[a}F_{bc]} = 0$.  
In fact, incorporate constraint damping by adding two fields, $\Psi$ 
and $\Phi$, to the EM equations.  These equations thus become   
$$\eqalign{  
\nabla^b \bigl( F_{ba} + g_{ba} \Psi \bigr) & = \kappa_1 n_a \Psi + 2\alpha_0 \nabla^b \phi F_{ba} \cr  
\nabla^b \bigl( (*F)_{ba} + g_{ba} \Phi \bigr) & = \kappa_2 n_a \Phi \cr  
}$$
where the $\kappa$s are real, positive constants.  In analogy to standard EM, 
a 4-current can be identified, namely, $J_a = 2\alpha_0 F_{ab} \nabla^b \phi$.
Going to the usual $3+1$ form, on defining the quantities  
$$\eqalign{ 
g_{ab} & = h_{ab} - n_a n_b \cr 
\epsilon_{bcd} & = n^a \epsilon_{abcd} = \sqrt{h} \, \varepsilon_{bcd} = \sqrt{h} \, [bcd] \cr 
(*F)_{ab} & = {1\over2} \epsilon_{abcd} F^{cd} \cr 
\E_a & = h_a{}^b F_{bc} n^c \cr  
\B_a & = - h_a{}^b (*F)_{bc} n^c = {1\over2} \epsilon_{abc} F^{bc} \cr  
\F_{ab} & = \epsilon_{abc} \B^c \cr 
\Pi & = - n^a \nabla_a \phi , \cr 
}$$
the equations, in $3+1$ form, become 
$$\eqalign{ 
%
%%G_{ab} & = ``8\pi G" T_{ab} \cr  
G_{ab} = T_{ab} & = h_a{}^i h_b{}^j {\perp\!T}_{ij} - 2 n_{(a} h_{b)}{}^i \, j_i + n_a n_b \, \rho \cr 
  & = 2 D_a \phi \, D_b \phi + 2 e^{-2\alpha_0\phi} \, \bigl( \B_a \B_b  \, - \E_a \E_b  \bigr) 
\cr
 & \qquad 
    - h_{ab} \, \Bigl[ D^c\phi \, D_c\phi - \Pi^2 + V + e^{-2\alpha_0\phi} \, \bigl( \B_{c} \B^{c} - \E_c \E^c \bigr) 
\cr 
 & \qquad 
    + 2 \, n_{(a} h_{b)}{}^i \, \Bigl[ 2 \Pi \, D_i\phi + 2 e^{-2\alpha_0\phi} \epsilon_{icd} \E^c \B^d \Bigr] 
\cr 
 & \qquad 
    + n_a n_b \, \Bigl[ D_c\phi D^c \phi + \Pi^2 + V + e^{-2\alpha_0\phi} \bigl( \B_{c} \B^{c} + \E_c \E^c \bigr) \, \Bigr]  \cr 
%
}$$
where ${\perp\!\! T}_{ij}$, $j_i$ and $\rho$ are the projected stress 
tensor, the momentum flux and the energy density, respectively.   
$$\eqalign{ 
%
\partial_t \phi & = \beta^a \partial_a \phi - \alpha \Pi \cr 
%
\partial_t \Pi & = \beta^a \partial_a \Pi - D^a\bigl( \alpha D_a \phi \bigr) + \alpha K \Pi - \alpha_0 \, \alpha e^{-2\alpha_0\phi} \bigl[ \B_a\B^a - \E_a\E^a \bigr] + {\alpha\over2} \, {\partial V \over \partial \phi} \cr 
%
\partial_t \E^a & = \beta^i \partial_i \E^a - \E^b \partial_b \beta^a + \epsilon^{abc} D_b \bigl( \alpha \B_c \bigr) + \alpha K \E^a - 2 \alpha_0 \, \alpha \bigl[ \epsilon^{abc} D_b \phi \B_c + \Pi \E^a \bigr] - \alpha D^a \Psi \cr   
%
\partial_t \Psi & = \beta^a \partial_a \Psi - \alpha D_a \E^a + 2\alpha_0 \, \alpha D_a \phi \E^a - \kappa_1 \alpha \Psi \cr  
%
\partial_t \B^a & = \beta^i \partial_i \B^a - \B^b \partial_b \beta^a - \epsilon^{abc} D_b \bigl( \alpha \E_c \bigr) + \alpha K \B^a + \alpha D^a \Phi \cr 
%
\partial_t \Phi & = \beta^a \partial_a \Phi + \alpha D_a \B^a - \kappa_2 \alpha \Phi \cr
%
}$$

These (``matter") equations with respect to BSSN variables (e.g. 
$\tilde{\gamma}^{ab}$, $\chi$, $\tilde{\Gamma}^i$, etc) then become (the 
Einstein equations are given elsewhere)  
$$\eqalign{ 
%
\partial_t \phi & = \beta^a \partial_a \phi - \alpha \Pi \cr 
%
\partial_t \Pi & = \beta^a \partial_a \Pi - \chi \tilde{\gamma}^{ab} \bigl[ \alpha \partial_a \partial_b \phi + \partial_a \alpha \partial_b \phi \bigr] + \alpha \chi \Gamma^i \partial_i \phi + {1\over2} \alpha \tilde{\gamma}^{ab} \partial_a \phi \, \partial_b \chi + \alpha K \Pi 
  \cr 
  & \qquad 
     - \alpha_0 \, \alpha e^{-2\alpha_0\phi} \bigl[ \B_a\B^a - \E_a\E^a \bigr] + {\alpha\over2} \, {\partial V \over \partial \phi} \cr 
%
\partial_t \E^a & = \beta^i \partial_i \E^a - \E^i \partial_i \beta^a 
  %\cr
  %& \qquad 
      + \chi^{1/2} \varepsilon^{abc} \Bigl\{ \alpha \bigl[ \partial_b \tilde{\gamma}_{cd} \B^d + \tilde{\gamma}_{cd} \partial_b \B^d \bigr] + \tilde{\gamma}_{cd} \B^d \Bigl[ \partial_b \alpha - \alpha {\partial_b \chi \over \chi} \Bigr] \Bigr\} 
  \cr
  & \qquad 
      + \alpha K \E^a - 2 \alpha_0 \, \alpha \bigl[ \chi^{1/2} \varepsilon^{abc} \partial_b \phi \tilde{\gamma}_{cd} \B^d + \Pi \E^a \bigr] - \alpha \chi \tilde{\gamma}^{ab} \partial_b \Psi \cr   
%
\partial_t \Psi & = \beta^a \partial_a \Psi - \alpha \Bigl[ \partial_a \E^a - {3\over2} {\partial_a \chi \over \chi} \, \E^a \Bigr] + 2\alpha_0 \, \alpha \partial_a \phi \E^a - \kappa_1 \alpha \Psi \cr  
%
\partial_t \B^a & = \beta^i \partial_i \B^a - \B^b \partial_b \beta^a - \chi^{1/2} \varepsilon^{abc} \Bigl\{ \alpha \bigl[ \partial_b \tilde{\gamma}_{cd} \E^d + \tilde{\gamma}_{cd} \partial_b \E^d \bigr] + \tilde{\gamma}_{cd} \E^d \Bigl[ \partial_b \alpha - \alpha {\partial_b \chi \over \chi} \Bigr] \Bigr\}  
  \cr 
  & \qquad 
      + \alpha K \B^a + \alpha \chi \tilde{\gamma}^{ab} \partial_b \Phi \cr 
%
\partial_t \Phi & = \beta^a \partial_a \Phi + \alpha \Bigl[ \partial_a \B^a - {3\over2} {\partial_a \chi \over \chi} \B^a \Bigr] - \kappa_2 \alpha \Phi \cr
%
}$$

\vfil\eject

\bigskip
\centerline{\bf Spherically symmetric BH solutions}
\bigskip

\noindent {\bf Magnetically charged black hole ($\alpha_0=1$)} 

\smallskip

The solution to the static, spherically symmetric, $\alpha_0=1$ EMD equations 
with a regular event horizon and magnetic charge takes the form in
Schwarzschild-like coordinates:   
$$\eqalign{ 
ds^2 & = - \Bigl( 1 - {2M \over r} \Bigr) \, dt^2 + \Bigl( 1 - {2M \over r} \Bigr)^{-1} \, dr^2 + r \Bigl( r - {Q_m^2 e^{-2\phi_0} \over M} \Bigr) d\Omega^2 \cr
F_{\theta\phi} & = Q_m \sin\theta \cr 
e^{-2\phi} & = e^{-2\phi_0} \Bigl( 1 - {Q_m^2 e^{-2\phi_0} \over M r} \Bigr) \cr
}$$
where the constants, $M$, $Q_m$ and $\phi_0$ are the ADM mass, the magnetic 
charge, and the asymptotic value of the dilaton, respectively.  There is a 
curvature singularity at $r=Q_m^2 e^{-2\phi_0}/M$ and a regular horizon at 
$r=2M$ for $Q_m^2 < 2M^2 e^{2\phi_0}$ which becomes singular when this 
inequality is saturated (i.e. the extremal limit).  

\medskip

\noindent {\bf Electrically charged black hole ($\alpha_0=1$)}  

\smallskip

There is a discrete electromagnetic duality in this theory which leaves the 
equations of motion unchanged (though the action does change as the $F^2$ 
term picks up a minus sign).  The transformation (again assuming $\alpha_0=1$) 
is 
$$\eqalign{ 
F_{ab} & \rightarrow {1\over2} e^{-2\phi} \epsilon_{abcd} F^{cd} \cr 
\phi & \rightarrow -\phi \cr 
g_{ab} & \rightarrow g_{ab} \cr  
}$$
In this theory this amounts to a means of 
generating new solutions.  In particular, one can take the above magnetically 
charged solution and generate an electrically charged black hole solution 
that is also static and spherically symmetric.  The solution is 
$$\eqalign{ 
ds^2 & = - \Bigl( 1 - {2M \over r} \Bigr) \, dt^2 + \Bigl( 1 - {2M \over r} \Bigr)^{-1} \, dr^2 + r \Bigl( r - {Q_e^2 \, e^{2\phi_0} \over M} \Bigr) d\Omega^2 \cr
F_{tr} & = {Q_e \over r^2} \cr 
e^{2\phi} & = e^{2\phi_0} \Bigl( 1 - {Q_e^2 \, e^{2\phi_0} \over M r} \Bigr) \cr
}$$
where the constants, $M$, $Q_e$ and $\phi_0$ are the ADM mass, the electric 
charge, and the asymptotic value of the dilaton, respectively.  There is a 
curvature singularity at $r=Q_e^2 e^{2\phi_0}/M$ and a regular horizon at 
$r=2M$ which becomes singular in the extremal limit, 
$Q_e^2 = 2M^2 e^{-2\phi_0}$.


\medskip

\noindent {\bf Magnetically charged black hole} 

\smallskip

Another generalization to consider is the case of these same black holes 
across theories, i.e. with general $\alpha_0$.  In this case, the solution 
to the static, spherically symmetric, EMD equations 
with a regular event horizon and magnetic charge takes the form (with $\beta \equiv 2\alpha_0^2 / (1+\alpha_0^2)$)  
$$\eqalign{ 
ds^2 & = - \Bigl( 1 - {r_{+} \over r} \Bigr) \Bigl( 1 - {r_{-} \over r} \Bigr)^{1-\beta} \, dt^2 + \Bigl( 1 - {r_{+} \over r} \Bigr)^{-1} \Bigl( 1 - {r_{-} \over r} \Bigr)^{\beta-1} \, dr^2 + r^2 \Bigl( 1 - {r_{-} \over r} \Bigr)^{\beta} d\Omega^2 \cr
F_{\theta\phi} & = Q_m \sin\theta \cr 
e^{-2\alpha_0\phi} & = e^{-2\alpha_0\phi_0} \, \Bigl( 1 - {r_{-} \over r} \Bigr)^{\beta} \cr
}$$
where $r_{+}$ and $r_{-}$ give the outer (event) and inner horizons and their 
combinations give the ADM mass, $M$, magnetic charge, $Q_m$, and asymptotic 
dilaton value, $\phi_0$, as follows:
$$\eqalign{ 
2M & = r_{+} + (1-\beta) \, r_{-}\cr 
2 Q_m^2 & = e^{2\alpha_0 \phi_0} \, r_{+} r_{-} (2 - \beta) \cr 
}$$
Note that for $\alpha_0=0$ ($\beta=0$) this reduces to Reissner-Nordstrom with
a constant dilaton while $\alpha_0=1$ ($\beta=1$) reproduces the above magnetic
solution for low energy string theory.  
%There is a 
%curvature singularity at $r=Q_m^2 e^{-2\phi_0}/M$, a regular horizon at 
%$r=2M$ for $Q_m^2 < 2M^2 e^{2\phi_0}$ which becomes singular when this 
%inequality is saturated (i.e. the extremal limit).  

\medskip

\noindent {\bf Electrically charged black hole}  

\smallskip

We can again use a discrete electromagnetic duality to generate the 
electrially charged solution. 
The transformation is 
$$\eqalign{ 
F_{ab} & \rightarrow {1\over2} e^{-2 \alpha_0 \phi} \epsilon_{abcd} F^{cd} \cr 
\phi & \rightarrow -\phi \cr 
g_{ab} & \rightarrow g_{ab} \cr  
}$$
Taking the above magnetically charged solution we get 
$$\eqalign{ 
ds^2 & = - \Bigl( 1 - {r_{+} \over r} \Bigr) \Bigl( 1 - {r_{-} \over r} \Bigr)^{1-\beta} \, dt^2 + \Bigl( 1 - {r_{+} \over r} \Bigr)^{-1} \Bigl( 1 - {r_{-} \over r} \Bigr)^{\beta-1} \, dr^2 + r^2 \Bigl( 1 - {r_{-} \over r} \Bigr)^{\beta} d\Omega^2 \cr
F_{tr} & = {Q_e \over r^2} \cr 
e^{2\alpha_0\phi} & = e^{2\alpha_0\phi_0} \, \Bigl( 1 - {r_{-} \over r} \Bigr)^{\beta} \cr
}$$
where, again, $r_{+}$ and $r_{-}$ give the outer (event) and inner horizons 
and their 
combinations give the ADM mass, $M$, electric charge, $Q_e$, and asymptotic 
dilaton value, $\phi_0$, as follows:
$$\eqalign{ 
2M & = r_{+} + (1-\beta) \, r_{-}\cr 
2 Q_e^2 & = e^{-2\alpha_0 \phi_0} \, r_{+} r_{-} (2 - \beta) \cr 
}$$

\medskip

\noindent {\bf Black holes in isotropic coordinates} 

\smallskip

Isotropic coordinates are useful with which to express the black hole 
solutions.  To this end, we find a new radial coordinate, ${\bar r}$, 
such that 
$$
{d {\bar r} \over {\bar r} } = {dr \over \sqrt{ (r - r_{+}) ( r - r_{-} ) } } 
$$
On integrating, and letting ${\bar r}$ approach $r$ at spatial infinity, 
we get 
$$
r = {1 \over {\bar r}} \, \biggl[ \Bigl( {\bar r} + {r_{+} + r_{-} \over 4} \Bigr)^2 - {r_{+} r_{-} \over 4} \biggr] 
$$
where we have 
$$\eqalign{ 
r_{+} & = M \, \biggl\{ 1 + \Bigl[ 1 - \bigl(1-\alpha_0^2\bigr) \, {Q^2 \over M^2} \Bigr]^{1/2} \, \biggr\} \cr 
r_{-} & = {Q^2 \over M} \, \bigl(1+\alpha_0^2\bigr) \, \biggl\{ 1 + \Bigl[ 1 - \bigl(1-\alpha_0^2\bigr) {Q^2 \over M^2} \, \Bigr]^{1/2} \, \biggr\}^{-1}  \cr 
}$$
where $Q^2=Q_m^2 e^{-2\alpha_0\phi_0}$ in the magnetic case and $Q^2 = Q_e^2 e^{2\alpha_0\phi_0}$ in the electric case. 

With this isotropic, radial coordinate, the metric for both magnetic and 
electric solutions takes the form 
$$\eqalign{ 
ds^2 & = - \alpha^2 dt^2 + \chi^{-1} 
    \bigl[ d{\bar r}^2 + {\bar r}^2 \, d\Omega^2 \bigr]  \cr  
  & = - { \bigl( {\bar r} - {\bar r}_H \bigr)^2 \bigl( {\bar r} + {\bar r}_H \bigr)^{2(1-\beta)} \over \bigl( {\bar r} + {\bar r}_1 \bigr)^{2-\beta} \bigl( {\bar r} + {\bar r}_2 \bigr)^{2-\beta} } \, dt^2 
  %\cr 
  %& \qquad 
    + {1\over {\bar r}^4} \bigl( {\bar r} + {\bar r}_1 \bigr)^{2-\beta} \bigl( {\bar r} + {\bar r}_2 \bigr)^{2-\beta} \bigl( {\bar r} + {\bar r}_H \bigr)^{2\beta} \, \bigl[ d{\bar r}^2 + {\bar r}^2 \, d\Omega^2 \bigr]  \cr  
}$$ 
where we have defined 
$$\eqalign{ 
{\bar r}_1 & = {1\over4} \bigl( \sqrt{r_{+}} - \sqrt{r_{-}} \bigr)^2 \cr 
{\bar r}_2 & = {1\over4} \bigl( \sqrt{r_{+}} + \sqrt{r_{-}} \bigr)^2 \cr 
{\bar r}_H & = {1\over4} \bigl( r_{+} - r_{-} \bigr) \cr 
}$$
with ${\bar r}_H$ the radial location of the horizon in these coordinates.  

In the magnetic case, the EM and dilaton fields take 
the form 
$$\eqalign{ 
F_{\theta\phi} & = Q_m \sin\theta \cr 
\B^{\bar r} & = Q_m \, { {\bar r}^4 \over \bigl( {\bar r} + {\bar r}_1 \bigr)^3 \bigl( {\bar r} + {\bar r}_2 \bigr)^3 } \, \Biggl[ { \bigl( {\bar r} + {\bar r}_1 \bigr) \bigl( {\bar r} + {\bar r}_2 \bigr) \over \bigl( {\bar r} + {\bar r}_H \bigr)^2 } \Biggr]^{3\beta/2} = {Q_m \over {\bar r}^2} \, \chi^{3/2} \cr 
e^{-2\alpha_0 \phi} & = e^{-2\alpha_0 \phi_0} \, { \bigl( {\bar r} + {\bar r}_H \bigr)^{2\beta} \over \bigl( {\bar r} + {\bar r}_1 \bigr)^\beta \bigl( {\bar r} + {\bar r}_2 \bigr)^\beta }  \cr 
}$$
In the electric case, the EM and dilaton fields take the form 
$$\eqalign{ 
F_{t{\bar r}} & = Q_e \, { \bigl( {\bar r}^2 - {\bar r}_H^2 \bigr) \over \bigl( {\bar r} + {\bar r}_1 \bigr)^2 \bigl( {\bar r} + {\bar r}_2 \bigr)^2 } \cr 
\E^{\bar r} & = - Q_e \, { {\bar r}^4 \over \bigl( {\bar r} + {\bar r}_1 \bigr)^3 \bigl( {\bar r} + {\bar r}_2 \bigr)^3 } \, \Biggl[ { \bigl( {\bar r} + {\bar r}_1 \bigr) \bigl( {\bar r} + {\bar r}_2 \bigr) \over \bigl( {\bar r} + {\bar r}_H \bigr)^2 } \Biggr]^{\beta/2} = - Q_e \, { {\bar r}^2 \over ({\bar r} + {\bar r}_1)^2 ({\bar r} + {\bar r}_2)^2 } \, \chi^{1/2} \cr 
e^{2\alpha_0 \phi} & = e^{2\alpha_0 \phi_0} \, { \bigl( {\bar r} + {\bar r}_H \bigr)^{2\beta} \over \bigl( {\bar r} + {\bar r}_1 \bigr)^\beta \bigl( {\bar r} + {\bar r}_2 \bigr)^\beta }  \cr 
}$$




\vfil\eject\end
